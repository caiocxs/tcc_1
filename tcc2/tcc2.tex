% ---------------------------------------------------------------
% TEMPLATE TCC2 - SENAC SANTO AMARO
% ---------------------------------------------------------------
% Trabalho de Conclusão de Curso 2
% ---------------------------------------------------------------

\documentclass[
    12pt,
    openright,
    oneside,
    a4paper,
    english,
    brazil
]{abntex2}

\usepackage{lmodern}
\usepackage[T1]{fontenc}
\usepackage[utf8]{inputenc}
\usepackage{lastpage}
\usepackage{indentfirst}
\usepackage{color}
\usepackage{graphicx}
\usepackage{microtype}
\usepackage{float}
\usepackage{booktabs}
\usepackage{multirow}
\usepackage{longtable}
\usepackage{amsmath,amssymb}
\usepackage[brazilian,hyperpageref]{backref}
\usepackage[alf]{abntex2cite}

\definecolor{blue}{RGB}{41,5,195}

\makeatletter
\hypersetup{
    pdftitle={\@title},
    pdfauthor={\@author},
    pdfsubject={TCC2 - SENAC Santo Amaro},
    pdfcreator={LaTeX with abnTeX2},
    pdfkeywords={tcc2}{senac}{santo amaro}{trabalho de conclusão},
    colorlinks=true,
    linkcolor=blue,
    citecolor=blue,
    filecolor=magenta,
    urlcolor=blue,
    bookmarksdepth=4
}
\makeatother

\titulo{Título do TCC2}
\autor{Nome Completo do Aluno}
\local{São Paulo}
\data{2025}
\orientador{Prof. Nome do Orientador}
\tipotrabalho{Trabalho de Conclusão de Curso II}
\instituicao{%
  SENAC -- Serviço Nacional de Aprendizagem Comercial
  \par
  Unidade Santo Amaro
  \par
  Curso de Bacharelado em Ciência da Computação
}
\preambulo{Trabalho de Conclusão de Curso II apresentado ao SENAC Santo Amaro como requisito para conclusão do curso de Bacharelado em Ciência da Computação.}

\makeindex

\begin{document}

\selectlanguage{brazil}
\frenchspacing

\pretextual

\imprimircapa
\imprimirfolhaderosto

\begin{resumo}
Este documento apresenta o TCC2 com base na estrutura recomendada por Raul Sidnei Wazlawick em \textit{Metodologia de Pesquisa para Ciência da Computação}. O foco está em evidenciar problema, objetivos, método, desenvolvimento, resultados e contribuição, evitando textos meramente descritivos.
\end{resumo}

\begin{resumo}[Abstract]
This document follows the TCC structure proposed by Wazlawick for Computer Science, emphasizing the research problem, objectives, proposed method, development details, and scientific analysis of results.
\end{resumo}

\tableofcontents*
\cleardoublepage

\textual

% ---------------------------------------------------------------
% INTRODUÇÃO
% ---------------------------------------------------------------
\chapter{Introdução}

\section{Contextualização do problema}
Descreva o cenário atual e a lacuna identificada, embasando o texto em dados, estatísticas, políticas ou práticas profissionais. Evite narrativas históricas extensas e mantenha o foco no problema.

Para citar a orientação do autor, use `\citeonline{Wazlawick2020}` quando precisar reproduzir uma afirmação direta sobre metodologia. Ao resumir outros trabalhos, integre as ideias ao seu texto e utilize `\cite{Silva2022}` como exemplo de citação indireta.

\section{Formulação do problema}
Coloque o problema em forma de pergunta clara, delimitando o que será investigado e qual resultado se espera atingir dentro da Computação.

-\section{Objetivos e hipótese}
\subsection{Objetivo geral}
Declare o propósito principal do TCC2 em uma frase direta, conectando-o ao problema e deixando claro se o foco é compreender, comparar, validar ou propor algo novo.

\subsection{Objetivos específicos}
Liste objetivos mensuráveis que estruturam o plano de trabalho. Tente utilizar verbos como revisar, comparar, modelar, quantificar ou validar. Exemplos:
\begin{itemize}
    \item Revisar obras-chave sobre [tema] e identificar lacunas;
    \item Modelar ou prototipar a abordagem proposta;
    \item Implementar experimentos controlados;
    \item Avaliar o desempenho por métricas (ex.: acurácia, latência).
\end{itemize}

\subsection{Hipótese ou critério de validação}
Descreva a hipótese que será testada ou o critério usado para afirmar que o objetivo geral foi atingido. Caso não haja hipótese formal, explique como as métricas escolhidas demonstrarão o sucesso esperado.

\section{Justificativa}
Ressalte a importância do problema para a Computação, destacando contribuições acadêmicas, sociais ou industriais. Explique como a proposta evita se tornar um “manual de ferramenta”.

% ---------------------------------------------------------------
% REVISÃO BIBLIOGRÁFICA
% ---------------------------------------------------------------
\chapter{Revisão Bibliográfica / Trabalhos Relacionados}

\section{Fundamentos teóricos necessários}
Explique os conceitos essenciais, definindo termos e equações quando necessário. Utilize blocos de texto ou pequenos quadros comparativos para sintetizar modelos.

Um exemplo de citação direta poderia ser `\citeonline{Silva2022}` para reforçar que determinados métodos exigem validação empírica; uma citação indireta pode ser escrita como “vários estudos destacam ...” seguido de `\cite{Wazlawick2020}`.

\section{Técnicas, métodos ou abordagens existentes}
Apresente abordagens já empregadas no contexto e destaque o que funcionou ou falhou. Inclua uma tabela comparativa simples para clarificar diferenças:
\begin{table}[htb]
  \centering
  \caption{Comparativo de abordagens relacionadas}
  \begin{tabular}{p{4cm}p{3cm}p{4cm}}
    \toprule
    Abordagem & Domínio & Limitação principal \\
    \midrule
    Técnica A & IoT & Falta de validação em larga escala \\
    Técnica B & Redes neurais & Alto custo computacional \\
    \bottomrule
  \end{tabular}
  \legend{Tabela ilustrativa com limitações observadas.}
\end{table}

\section{Trabalhos correlatos e limitações}
Liste artigos diretamente relacionados e explique por que cada um não resolve totalmente o problema (ex.: falta de comparação, escopo diferente, dados limitados). Relacione essas lacunas com o que sua pesquisa pretende preencher.

\textit{Evite a “síndrome da intersecção esquecida”: inclua trabalhos que aplicaram a técnica na mesma área.}

% ---------------------------------------------------------------
% METODOLOGIA
% ---------------------------------------------------------------
\chapter{Metodologia}

\section{Tipo de pesquisa e hipótese}
Informe qual tipo de pesquisa será conduzido (exploratória, empírica, formal etc.) e registre as hipóteses que serão avaliadas. Caso não haja hipótese formal, descreva o critério de sucesso esperado.

\section{Procedimentos da abordagem proposta}
Detalhe etapas da metodologia em ordem cronológica. Use listas numeradas, diagramas de fluxo ou pseudocódigo para tornar o processo claro.

\section{Ferramentas, tecnologias e ambientes}
Liste softwares, linguagens, bibliotecas, frameworks e ambientes (laboratórios, nuvem, simulação) que serão usados. Justifique a escolha com base no problema e nas métricas.

\section{Procedimentos experimentais e métricas}
Explique como os experimentos serão configurados, quais variáveis serão controladas e quais métricas demonstram que a proposta cumpre os objetivos (ex.: acurácia, latência, custo). Inclua tabelas que antecipem o plano de coleta, por exemplo:
\begin{table}[htb]
  \centering
  \caption{Plano de experimentação}
  \begin{tabular}{lccc}
    \toprule
    Experimento & Variável controlada & Métrica principal \\
    \midrule
    Comparação com baseline & algoritmo & acurácia \\
    Escalabilidade & carga & tempo médio \\
    \bottomrule
  \end{tabular}
  \legend{Exemplo de como documentar experimentos.}
\end{table}

\textit{Propor algo não basta: demonstre como será validado.}

% ---------------------------------------------------------------
% DESENVOLVIMENTO
% ---------------------------------------------------------------
\chapter{Desenvolvimento / Implementação / Experimentos}

\section{Modelagem e arquitetura}
Apresente diagramas (UML, blocos ou fluxogramas) que mostram como os módulos se relacionam, identificando componentes críticos e como eles atendem ao problema e à metodologia.

\section{Detalhes de implementação}
Explique escolhas tecnológicas, estruturas de dados e integrações. Inclua trechos de código relevantes ou pseudoexemplos se esclarecerem decisões de pesquisa.

\section{Experimentos e cenários de teste}
Descreva os cenários planejados, parâmetros testados e por que cada experimento está alinhado aos objetivos. Liste os conjuntos de dados, variações e número de repetições.

\section{Coleta e tratamento de dados}
Explique como os dados foram coletados (logs, sensores, APIs), armazenados e preparados (limpeza, normalização, anonimização). Registre ferramentas usadas no pré-processamento.

\textit{O desenvolvimento só é pesquisa se estiver claramente ligado ao problema e à metodologia definida.}

% ---------------------------------------------------------------
% RESULTADOS
% ---------------------------------------------------------------
\chapter{Resultados e Análise}

\section{Apresentação dos resultados}
Mostre resultados com tabelas, gráficos e figuras. Diferencie:
\begin{itemize}
  \item \textbf{Figura:} imagens, diagramas conceituais ou arquiteturais que ilustram estrutura ou fluxo.
  \item \textbf{Gráfico:} plot de dados (linhas, barras, boxplots) que compara métricas obtidas.
\end{itemize}

Exemplo de tabela:
\begin{table}[htb]
  \centering
  \caption{Comparação de métricas reportadas}
  \begin{tabular}{lcc}
    \toprule
    Método & Acurácia & Tempo médio (ms) \\
    \midrule
    Proposta & 92\% & 180 \\
    Baseline & 88\% & 220 \\
    \bottomrule
  \end{tabular}
  \legend{Tabela exemplificando como comparar métricas.}
\end{table}

Ao inserir um gráfico (por exemplo `assets/figures/desempenho.png`), use:
\begin{verbatim}
\begin{figure}[htb]
  \centering
  \includegraphics[width=0.8\textwidth]{assets/figures/desempenho.png}
  \caption{Gráfico de comparação das métricas de acurácia.}
  \label{fig:grafico-desempenho}
\end{figure}
\end{verbatim}

\section{Análise crítica e comparação}
Discuta as descobertas com base na revisão bibliográfica e na hipótese. Evidencie ganhos e perdas.

\section{Discussão}
Relacione os resultados aos objetivos e à hipótese, explicitando se o critério de sucesso foi atingido.

\textit{Resultados devem ser analisados, não apenas exibidos.}

% ---------------------------------------------------------------
% CONCLUSÃO
% ---------------------------------------------------------------
\chapter{Conclusão}

\section{Retomada do problema e dos objetivos}
Retome o problema e destaque o que foi atingido.

\section{Contribuições para a Computação}
Explique o que a pesquisa trouxe de novo.

\section{Limitações e trabalhos futuros}
Liste limites e sugira próximos passos coerentes.

% ---------------------------------------------------------------
% OBSERVAÇÕES DO AUTOR
% ---------------------------------------------------------------
\chapter*{Observações importantes}
\addcontentsline{toc}{chapter}{Observações importantes}

Para graduação, Wazlawick aceita trabalhos aplicados desde que:
\begin{itemize}
    \item O problema seja preciso;
    \item A solução esteja fundamentada;
    \item Haja análise crítica;
    \item Não seja apenas entrega funcional.
\end{itemize}

\postextual

\bibliography{bibliografia_tcc2}

\end{document}
