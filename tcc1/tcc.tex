% ---------------------------------------------------------------
% TEMPLATE TCC1 - SENAC SANTO AMARO
% ---------------------------------------------------------------
% Trabalho de Conclusão de Curso 1
% Elaborado para apresentação da proposta inicial do TCC
% ---------------------------------------------------------------

\documentclass[
    12pt,                % tamanho da fonte
    openright,           % capítulos começam em página ímpar
    oneside,             % para impressão em apenas um lado do papel
    a4paper,             % tamanho do papel
    english,             % idioma adicional
    brazil               % idioma principal
]{abntex2}

% ---------------------------------------------------------------
% PACOTES
% ---------------------------------------------------------------
\usepackage{lmodern}                % Fonte Latin Modern
\usepackage[T1]{fontenc}            % Seleção de códigos de fonte
\usepackage[utf8]{inputenc}         % Codificação do documento
\usepackage{lastpage}               % Usado pela Ficha catalográfica
\usepackage{indentfirst}            % Indenta o primeiro parágrafo
\usepackage{color}                  % Controle de cores
\usepackage{graphicx}               % Inclusão de gráficos
\usepackage{microtype}              % Melhorias de justificação
\usepackage{float}                  % Para posicionamento de figuras
\usepackage{booktabs}               % Tabelas profissionais
\usepackage{multirow}               % Células mescladas em tabelas
\usepackage{longtable}              % Tabelas longas
\usepackage{amsmath,amssymb}        % Símbolos matemáticos
\usepackage[brazilian,hyperpageref]{backref}  % Páginas com citações
\usepackage[alf]{abntex2cite}       % Citações ABNT

% ---------------------------------------------------------------
% CONFIGURAÇÕES DO PDF
% ---------------------------------------------------------------
\makeatletter
\hypersetup{
    pdftitle={\@title},
    pdfauthor={\@author},
    pdfsubject={TCC1 - SENAC Santo Amaro},
    pdfcreator={LaTeX with abnTeX2},
    pdfkeywords={tcc}{senac}{santo amaro}{trabalho de conclusão},
    colorlinks=true,        % false: links em caixas; true: links coloridos
    linkcolor=blue,         % cor dos links internos
    citecolor=blue,         % cor dos links para bibliografia
    filecolor=magenta,      % cor dos links para arquivos
    urlcolor=blue,
    bookmarksdepth=4
}
\makeatother

% ---------------------------------------------------------------
% CONFIGURAÇÕES DE APARÊNCIA DO PDF FINAL
% ---------------------------------------------------------------
\definecolor{blue}{RGB}{41,5,195}

% ---------------------------------------------------------------
% INFORMAÇÕES DO DOCUMENTO
% ---------------------------------------------------------------
\titulo{Título do Trabalho de Conclusão de Curso}
\autor{Nome Completo do Aluno}
\local{São Paulo}
\data{2025}
\orientador{Prof. Nome do Orientador}
% \coorientador{Prof. Nome do Coorientador} % Descomente se houver coorientador

\instituicao{%
  SENAC -- Serviço Nacional de Aprendizagem Comercial
  \par
  Unidade Santo Amaro
  \par
  Curso de [Nome do Curso]
}

\tipotrabalho{Trabalho de Conclusão de Curso I}

\preambulo{Proposta de Trabalho de Conclusão de Curso apresentado ao SENAC Santo Amaro como requisito parcial para aprovação na disciplina de TCC1 do curso de [Nome do Curso].}

% ---------------------------------------------------------------
% COMPILA O ÍNDICE
% ---------------------------------------------------------------
\makeindex

% ---------------------------------------------------------------
% INÍCIO DO DOCUMENTO
% ---------------------------------------------------------------
\begin{document}

% Seleciona o idioma do documento
\selectlanguage{brazil}

% Retira espaço extra obsoleto entre as frases
\frenchspacing

% ---------------------------------------------------------------
% ELEMENTOS PRÉ-TEXTUAIS
% ---------------------------------------------------------------
\pretextual

% ---------------------------------------------------------------
% CAPA
% ---------------------------------------------------------------
\imprimircapa

% ---------------------------------------------------------------
% FOLHA DE ROSTO
% ---------------------------------------------------------------
\imprimirfolhaderosto

% ---------------------------------------------------------------
% RESUMO
% ---------------------------------------------------------------
\begin{resumo}
Este documento apresenta a proposta inicial do Trabalho de Conclusão de Curso (TCC1) desenvolvido no SENAC Santo Amaro. O objetivo deste trabalho é [descrever brevemente o objetivo principal do TCC]. A metodologia a ser empregada inclui [mencionar a abordagem metodológica]. Como resultados esperados, pretende-se [descrever os principais resultados esperados]. Este trabalho está organizado em seções que apresentam a introdução ao tema, o referencial teórico que fundamenta a pesquisa, o plano de desenvolvimento detalhado e o cronograma de execução das atividades planejadas para o TCC2.

\vspace{\onelineskip}

\noindent
\textbf{Palavras-chave}: palavra1. palavra2. palavra3. palavra4. palavra5.
\end{resumo}

% ---------------------------------------------------------------
% LISTA DE ILUSTRAÇÕES
% ---------------------------------------------------------------
% \pdfbookmark[0]{\listfigurename}{lof}
% \listoffigures*
% \cleardoublepage

% ---------------------------------------------------------------
% LISTA DE TABELAS
% ---------------------------------------------------------------
% \pdfbookmark[0]{\listtablename}{lot}
% \listoftables*
% \cleardoublepage

% ---------------------------------------------------------------
% SUMÁRIO
% ---------------------------------------------------------------
\pdfbookmark[0]{\contentsname}{toc}
\tableofcontents*
\cleardoublepage

% ---------------------------------------------------------------
% ELEMENTOS TEXTUAIS
% ---------------------------------------------------------------
\textual

% ---------------------------------------------------------------
% INTRODUÇÃO
% ---------------------------------------------------------------
\chapter{Introdução}
\label{cap:introducao}

Baseada em \textit{Metodologia de Pesquisa para Ciência da Computação} (Wazlawick), esta introdução deixa explícitos o problema que será investigado, os objetivos que orientam o estudo, a justificativa para a escolha do tema e como o trabalho está organizado — tudo isso sem se perder em descrições históricas, pois o foco do TCC1 é planejar a pesquisa.

\section{Contextualização do problema}

Apresente o cenário que originou o problema, ressaltando fatos, indicadores e lacunas identificadas. Em TCC1, essa contextualização demonstra que o problema já foi mapeado e justifica sua importância para a Computação.

Use listas ou marcadores para organizar dados (ex.: políticas públicas, padrões da indústria ou pesquisas recentes) e sempre cite as fontes para mostrar que o problema já foi investigado.

Por exemplo, uma citação direta pode aparecer assim: segundo \citeonline{RCoreTeam2025}, a reproducibilidade em análises estatísticas depende de ambientes controlados. Já uma citação indireta pode relatar que estudos sobre “tidy data” vêm destacando boas práticas de organização dos dados \cite{Wickham2014}.

\section{Formulação do problema}

Defina o problema em forma de pergunta clara e objetiva. Em TCC1, já se espera um foco preciso: o problema deve apontar o que será investigado (o “o quê”), quem é impactado e a motivação da Computação por trás da investigação.

Delimite o escopo, deixando explícito o que está dentro do projeto e o que será deixado para etapas futuras, para evitar confundir tema com problema.

\section{Objetivos e hipótese}

Como o foco do TCC1 é o planejamento investigativo, esta seção precisa mostrar objetivos bem escolhidos (geral e específicos) e uma hipótese/critério de validação definido.

\subsection{Objetivo geral}

Declare em uma frase direta o ganho científico que se pretende alcançar (entendimento, comparação, validação ou melhoria). Vincule o objetivo ao problema para reforçar que o trabalho tem uma direção.

 \subsection{Objetivos específicos}

Liste objetivos mensuráveis que compõem o plano: exemplos incluem revisão crítica da literatura, definição de parâmetros e requisitos da hipótese, planejamento dos experimentos, e identificação das métricas.
\begin{itemize}
    \item Revisar criticamente o estado da arte sobre [tema];
    \item Identificar variáveis chave que influenciam o fenômeno estudado;
    \item Desenhar o esboço da metodologia e dos experimentos;
    \item Definir métricas de validação e critérios de sucesso.
\end{itemize}

\subsection{Hipótese ou critério de validação}

Descreva a hipótese principal que se pretende confrontar ou o critério que indicará o sucesso da pesquisa. Caso não haja hipótese formal, explique o que será preciso demonstrar para considerar o objetivo atingido.

Evite hipóteses vagas; formule algo plausível, mensurável e diretamente ligado aos objetivos geral e específicos.

\section{Justificativa}

Explicite por que resolver esse problema importa para a Computação, destacando relevância acadêmica ou profissional. Mostre como a justificativa está alinhada aos objetivos e à metodologia proposta, reforçando que o trabalho não é apenas um texto sobre tema popular.

%\section{Organização do trabalho}

%\textit{Erro comum apontado pelo autor: introduções longas e históricas que não deixam claro qual problema será resolvido.}

% ---------------------------------------------------------------
% REVISÃO BIBLIOGRÁFICA
% ---------------------------------------------------------------
\chapter{Revisão Bibliográfica / Trabalhos Relacionados}
\label{cap:revisao}

Organize este capítulo por conceitos, técnicas ou dimensões do problema, não simplesmente por autor. Use subtítulos para dividir as grandes áreas e construa uma linha de argumentação que sustente a lacuna abordada.

\section{Fundamentos teóricos necessários}

Explique os principais conceitos e teorias que ajudam a compreender o problema. Em TCC1, essa base teórica mostra que o problema foi mapeado e que há diretrizes para a metodologia futura.

Considere quadros ou tabelas resumindo definições quando fizer sentido.

Ao citar diretamente uma obra clássica, use o comando `\citeonline`, por exemplo: \citeonline{Ihaka1996} evidencia como a visualização e manipulação de dados permite demonstrar achados. Para citações indiretas, faça como em `\cite{Field2012}` e integre a informação ao seu texto.

\section{Técnicas, métodos ou abordagens existentes}

Apresente métodos e abordagens aplicados em contextos similares, destacando onde eles não atendem completamente ao problema formulado.

Uma tabela comparativa ajuda a ilustrar limitações:
\begin{table}[htb]
    \centering
    \caption{Comparação de abordagens relacionadas}
    \begin{tabular}{lcc}
        \toprule
        Abordagem & Aplicação & Limitação identificada \\
        \midrule
        Técnica A & domínio X & falta de validação em escala real \\
        Técnica B & domínio Y & alta complexidade de implantação \\
        \bottomrule
    \end{tabular}
    \legend{Exemplo para registrar limitações observadas.}
\end{table}

\section{Trabalhos correlatos e limitações identificadas}

Liste pesquisas que se aproximam do foco proposto e identifique lacunas que justificam o novo trabalho. Esse diagnóstico reforça o diferencial da proposta.

Inclua notas curtas sobre o que seu TCC1 pretende resolver em comparação com cada trabalho citado.

Inclua notas curtas sobre o que seu TCC1 pretende resolver em comparação com cada trabalho citado.

\textit{Cuidado com a “síndrome da intersecção esquecida”: revisar a técnica e a área, mas ignorar outros trabalhos que já aplicaram aquela técnica naquela área.}

% ---------------------------------------------------------------
% METODOLOGIA PROPOSTA
% ---------------------------------------------------------------
\chapter{Metodologia proposta}
\label{cap:metodologia}

Como TCC1, este capítulo deve deixar claro como a pesquisa será conduzida no futuro, mostrando que existe um plano consistente para verificar o problema.

\section{Tipo de pesquisa e hipótese de trabalho}

Indique se a pesquisa será exploratória, empírica, formal ou outro tipo, e apresente a hipótese ou suposições que serão avaliadas nos momentos subsequentes.

Caso ainda não exista hipótese formal, descreva a expectativa de resultado (critério de sucesso) e quais evidências serão necessárias.

\section{Método proposto ou abordagem adotada}

Descreva o fluxo metodológico previsto, incluindo etapas, frameworks ou diagramas que guiarão as atividades do TCC2.

Transforme o método em etapas numeradas (ex.: 1. Revisão aprofundada, 2. Modelagem da proposta, 3. Planejamento de experimentos) para facilitar o acompanhamento.

\section{Ferramentas, tecnologias e ambientes previstos}

Liste as ferramentas, linguagens e ambientes que se pretende usar, justificando por que elas são adequadas para validar a proposta.

Inclua observações sobre necessidade de infraestrutura, dados sensíveis ou licenças.

\section{Procedimentos experimentais e métricas de avaliação esperadas}

Explique como os experimentos serão desenhados, quais variáveis serão observadas e quais métricas permitirão responder ao problema e aos objetivos. Mesmo no TCC1, essa descrição mostra que a validação já foi pensada.

Use tabelas ou listas para vincular cada experimento a uma métrica (ex.: experimento de comparação -> acurácia, experimento de carga -> tempo médio).

\textit{Propor algo não é método suficiente; mostre como será possível verificar se a proposta atende ao problema e aos objetivos.}

% ---------------------------------------------------------------
% CRONOGRAMA E PLANEJAMENTO ESPERADO
% ---------------------------------------------------------------
\chapter{Cronograma e planejamento esperado}
\label{cap:cronograma}

Como o TCC1 ainda se concentra em planejar a pesquisa, este capítulo apresenta o cronograma de atividades previsto e o planejamento de entregas para o próximo período, mostrando que o método já está sendo coordenado.

\begin{table}[htb]
    \centering
    \caption{Cronograma de atividades do TCC1}
    \label{tab:cronograma}
    \begin{tabular}{l|c|c|c|c}
        \toprule
        \textbf{Atividade} & \textbf{Semana 1} & \textbf{Semana 2} & \textbf{Semana 3} & \textbf{Semana 4} \\
        \midrule
        Revisão bibliográfica inicial & X & X & & \\
        \midrule
        Refinamento da pergunta de pesquisa & & X & X & \\
        \midrule
        Proposição metodológica detalhada & & & X & X \\
        \midrule
        Definição de ferramentas e ambientes & & & X & \\
        \midrule
        Elaboração do relatório parcial (TCC1) & & & & X \\
        \bottomrule
    \end{tabular}
    \legend{Fonte: Elaborado pelo autor (2025)}
\end{table}

A tabela acima é um exemplo; ajuste as semanas, atividades e marcos conforme o calendário real da sua turma. Utilize “X” ou ✓ para indicar entregas previstas e acrescente colunas caso o semestre tenha mais semanas.

\section{Planejamento esperado}

\begin{itemize}
    \item Atualizar o referencial conforme novas leituras relevantes surgirem;
    \item Ajustar a metodologia proposta com feedback do orientador;
    \item Registrar decisões de ferramentas, dados e métricas que guiarão o TCC2;
    \item Preparar a entrega do relatório parcial com justificativas e próximos passos;
    \item Validar o cronograma com o calendário acadêmico do semestre.
\end{itemize}

\textit{No lugar da conclusão, o TCC1 deve explicitar como o TCC2 será conduzido.}

Use este capítulo também como referência para atualizações futuras: revise o cronograma a cada reunião com o orientador e anote alterações nas atividades/marcos.

% ---------------------------------------------------------------
% ELEMENTOS PÓS-TEXTUAIS
% ---------------------------------------------------------------
\postextual

% ---------------------------------------------------------------
% REFERÊNCIAS
% ---------------------------------------------------------------
\bibliography{bibliografia}

% ---------------------------------------------------------------
% APÊNDICES (se necessário)
% ---------------------------------------------------------------
% \begin{apendicesenv}
% \partapendices
%
% \chapter{Nome do Apêndice}
% \label{ap:apendice1}
%
% Conteúdo do apêndice (material elaborado pelo próprio autor).
%
% \end{apendicesenv}

% ---------------------------------------------------------------
% ANEXOS (se necessário)
% ---------------------------------------------------------------
% \begin{anexosenv}
% \partanexos
%
% \chapter{Nome do Anexo}
% \label{an:anexo1}
%
% Conteúdo do anexo (material de terceiros).
%
% \end{anexosenv}

\end{document}
